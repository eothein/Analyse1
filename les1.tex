\chapter{Introductie}

\section{Algemeen}
Dit zijn de cursusnota's bij het vak Analyse 1, voor de afstandsstudenten. In dit document ga je wat extra uitleg vinden die de slides moeten verduidelijken. 

De bedoeling is dat je de stof zelfstandig tot je zelf neemt en de oefeningen maakt per hoofdstuk. De oplossingen van uw oefeningen post je op het forum van het vak en worden besproken door alle studenten (en natuurlijk door de docent). 

Voor de oefeningen zal gebruik gemaakt worden van Visual paradigm, wat je kan downloaden op 
%
\url{http://www.visual-paradigm.com/download/}

Doorheen de cursusnota's vind je enkele opdrachten. De bedoeling is de opdrachten te bespreken op het forum. 

\section{Analyse}
Het vak analyse bereidt de studenten voor tot informatici die niet enkel en alleen verschillende soorten programmeertalen machtig zijn, maar ook weten welke methodieken gebruikt kunnen worden om een ICT project succesvol af te leveren. Factoren die het al dan niet falen van een project be\"invloeden zijn onder andere:

\begin{itemize}
	\item Tijdsbestek
	\item Financieel kader
	\item Manpower
	\item \dots
\end{itemize}

\begin{exercise}
	Welke eigenschappen en factoren kunnen nog een invloed hebben op het al dan niet falen van een ICT project? Welke zijn minder relevant?
\end{exercise}

De realiteit echter, toont aan dat maar al te vaak ICT projecten falen. Voorbeelden vind je wel overal zoals bijvoorbeeld het annuleren van een ERP (Enterprise Resource planning) project bij de luchtmacht van de verenigde staten nadat al 1 biljoen dollar gespendeerd was \cite{Kanaracus2012} \cite{failedProjects}.

\begin{exercise}
	Kan je enkele Belgische of Vlaamse ICT projecten vinden die onlangs gefaald zijn?
\end{exercise}

De auteurs van \cite{International1995} vergelijken het maken van software  met het bouwen van een brug. Ze vermelden dat bruggen vaak succesvol gebouwd worden doordat er veel details aanwezig zijn en dat het volledig beschreven is volgens bepaalde specificaties. Indien men dezelfde methodologie zou gebruiken voor het maken van software zou men wel eens bedrogen kunnen uitkomen. Het rapport toont ook aan dat maar 26\% van gestarte ICT projecten succesvol en binnen het tijdbestek en financieel plan afgeleverd werden. De rest van de projecten werden ofwel niet afgeleverd ofwel voldeden ze niet aan de eisen van de opdrachtgever. Nu moet dit rapport wel met de nodige kritische geest gelezen worden. Zier hiervoor bijvoorbeeld \cite{Glass2006}. Niettemin, is het wel een indicatie dat een softwareproject op een degelijke manier moet aangepakt worden. 

Het is in de lessen analyse dat we hiervoor de fundamenten zullen aanbrengen: kwaliteitsvolle ICT projecten opleveren.



