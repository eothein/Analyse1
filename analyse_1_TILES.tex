\documentclass[11pt,fleqn]{book}

%%%%%%%%%%%%%%%%%%%%%%%%%%%%%%%%%%%%%%%%%
% The Legrand Orange Book
% Structural Definitions File
% Version 2.0 (9/2/15)
%
% Original author:
% Mathias Legrand (legrand.mathias@gmail.com) with modifications by:
% Vel (vel@latextemplates.com)
% 
% This file has been downloaded from:
% http://www.LaTeXTemplates.com
%
% License:
% CC BY-NC-SA 3.0 (http://creativecommons.org/licenses/by-nc-sa/3.0/)
%
%%%%%%%%%%%%%%%%%%%%%%%%%%%%%%%%%%%%%%%%%

%----------------------------------------------------------------------------------------
%	VARIOUS REQUIRED PACKAGES AND CONFIGURATIONS
%----------------------------------------------------------------------------------------

\usepackage{pgfplotstable}
\usepackage{pgfplots}
\usepackage{tikz}
\usepackage[top=3cm,bottom=3cm,left=3cm,right=3cm,headsep=10pt,a4paper]{geometry} % Page margins
\usepackage{graphicx} % Required for including pictures
\graphicspath{{./Pictures/}, {./images/}} % Specifies the directory where pictures are stored
\usepackage{tikz} % Required for drawing custom shapes
\usepackage[english,dutch]{babel} % English language/hyphenation
\usepackage{paralist}
\usepackage{enumitem} % Customize lists
\setlist{nolistsep} % Reduce spacing between bullet points and numbered lists
\usepackage{booktabs} % Required for nicer horizontal rules in tables
\usepackage{xcolor} % Required for specifying colors by name
\usepackage{hyperref}
\definecolor{hogentblue}{RGB}{0,111,184} % Define the orange color used for highlighting throughout the book



%----------------------------------------------------------------------------------------
%	FONTS
%----------------------------------------------------------------------------------------

\usepackage{avant} % Use the Avantgarde font for headings
%\usepackage{times} % Use the Times font for headings
\usepackage{mathptmx} % Use the Adobe Times Roman as the default text font together with math symbols from the Sym­bol, Chancery and Com­puter Modern fonts
\usepackage{microtype} % Slightly tweak font spacing for aesthetics
\usepackage[utf8]{inputenc} % Required for including letters with accents
\usepackage[T1]{fontenc} % Use 8-bit encoding that has 256 glyphs

%----------------------------------------------------------------------------------------
%	BIBLIOGRAPHY AND INDEX
%----------------------------------------------------------------------------------------


\usepackage{calc} % For simpler calculation - used for spacing the index letter headings correctly
\usepackage{makeidx} % Required to make an index
\makeindex % Tells LaTeX to create the files required for indexing

%----------------------------------------------------------------------------------------
%	MAIN TABLE OF CONTENTS
%----------------------------------------------------------------------------------------

\usepackage{titletoc} % Required for manipulating the table of contents

% Chapter text styling
\titlecontents{chapter}[1.25cm] % Indentation
{\addvspace{12pt}\large\sffamily\bfseries} % Spacing and font options for chapters
{\color{hogentblue!60}\contentslabel[\Large\thecontentslabel]{1.25cm}\color{hogentblue}} % Chapter number
{\color{hogentblue}}  
{\color{hogentblue!60}\normalsize\;\titlerule*[.5pc]{.}\;\thecontentspage} % Page number

% Section text styling
\titlecontents{section}[1.25cm] % Indentation
{\addvspace{3pt}\sffamily\bfseries} % Spacing and font options for sections
{\contentslabel[\thecontentslabel]{1.25cm}} % Section number
{}
{\hfill\color{black}\thecontentspage} % Page number
[]

% Subsection text styling
\titlecontents{subsection}[1.25cm] % Indentation
{\addvspace{1pt}\sffamily\small} % Spacing and font options for subsections
{\contentslabel[\thecontentslabel]{1.25cm}} % Subsection number
{}
{\ \titlerule*[.5pc]{.}\;\thecontentspage} % Page number
[]

% Chapter text styling
\titlecontents{lchapter}[0em] % Indenting
{\addvspace{15pt}\large\sffamily\bfseries} % Spacing and font options for chapters
{\color{hogentblue}\contentslabel[\Large\thecontentslabel]{1.25cm}\color{ocre}} % Chapter number
{}  
{\color{hogentblue}\normalsize\sffamily\bfseries\;\titlerule*[.5pc]{.}\;\thecontentspage} % Page number

% Section text styling
\titlecontents{lsection}[0em] % Indenting
{\sffamily\small} % Spacing and font options for sections
{\contentslabel[\thecontentslabel]{1.25cm}} % Section number
{}
{}

% Subsection text styling
\titlecontents{lsubsection}[.5em] % Indentation
{\normalfont\footnotesize\sffamily} % Font settings
{}
{}
{}
\usepackage{fancyhdr} % Required for header and footer configuration
\pagestyle{fancy}
\renewcommand{\chaptermark}[1]{\markboth{\sffamily\normalsize\bfseries\chaptername\ \thechapter.\ #1}{}} % Chapter text font settings
\renewcommand{\sectionmark}[1]{\markright{\sffamily\normalsize\thesection\hspace{5pt}#1}{}} % Section text font settings
\fancyhf{} \fancyhead[LE,RO]{\sffamily\normalsize\thepage} % Font setting for the page number in the header
\fancyhead[LO]{\rightmark} % Print the nearest section name on the left side of odd pages
\fancyhead[RE]{\leftmark} % Print the current chapter name on the right side of even pages
\renewcommand{\headrulewidth}{0.5pt} % Width of the rule under the header
\addtolength{\headheight}{2.5pt} % Increase the spacing around the header slightly
\renewcommand{\footrulewidth}{0pt} % Removes the rule in the footer
\fancypagestyle{plain}{\fancyhead{}\renewcommand{\headrulewidth}{0pt}} % Style for when a plain pagestyle is specified

% Removes the header from odd empty pages at the end of chapters
\makeatletter
\renewcommand{\cleardoublepage}{
\clearpage\ifodd\c@page\else
\hbox{}
\vspace*{\fill}
\thispagestyle{empty}
\newpage
\fi}

\makeatletter
\renewcommand{\@seccntformat}[1]{\llap{\textcolor{hogentblue}{\csname the#1\endcsname}\hspace{1em}}}                    
\renewcommand{\section}{\@startsection{section}{1}{\z@}
{-4ex \@plus -1ex \@minus -.4ex}
{1ex \@plus.2ex }
{\normalfont\large\sffamily\bfseries}}
\renewcommand{\subsection}{\@startsection {subsection}{2}{\z@}
{-3ex \@plus -0.1ex \@minus -.4ex}
{0.5ex \@plus.2ex }
{\normalfont\sffamily\bfseries}}
\renewcommand{\subsubsection}{\@startsection {subsubsection}{3}{\z@}
{-2ex \@plus -0.1ex \@minus -.2ex}
{.2ex \@plus.2ex }
{\normalfont\small\sffamily\bfseries}}                        
\renewcommand\paragraph{\@startsection{paragraph}{4}{\z@}
{-2ex \@plus-.2ex \@minus .2ex}
{.1ex}
{\normalfont\small\sffamily\bfseries}}


\newcommand{\chapterimage}[1]{\renewcommand{\thechapterimage}{#1}}%
\def\@makechapterhead#1{%
{\parindent \z@ \raggedright \normalfont
\ifnum \c@secnumdepth >\m@ne
\if@mainmatter
\begin{tikzpicture}[remember picture,overlay]
\node at (current page.north west)
{\begin{tikzpicture}[remember picture,overlay]
\draw[anchor=west] (\Gm@lmargin,-9cm) node [line width=2pt,rounded corners=15pt,draw=hogentblue,fill=white,fill opacity=0.5,inner sep=15pt]{\strut\makebox[22cm]{}};
\draw[anchor=west] (\Gm@lmargin+.3cm,-9cm) node {\huge\sffamily\bfseries\color{black}\thechapter. #1\strut};
\end{tikzpicture}};
\end{tikzpicture}
\else
\begin{tikzpicture}[remember picture,overlay]
\node at (current page.north west)
{\begin{tikzpicture}[remember picture,overlay]
\node[anchor=north west,inner sep=0pt] at (0,0) {\includegraphics[width=\paperwidth]{\thechapterimage}};
\draw[anchor=west] (\Gm@lmargin,-9cm) node [line width=2pt,rounded corners=15pt,draw=hogentblue,fill=white,fill opacity=0.5,inner sep=15pt]{\strut\makebox[22cm]{}};
\draw[anchor=west] (\Gm@lmargin+.3cm,-9cm) node {\huge\sffamily\bfseries\color{black}#1\strut};
\end{tikzpicture}};
\end{tikzpicture}
\fi\fi\par\vspace*{270\p@}}}

%----------------------------------------------------------------------------------------
%	Symboldefinitions
%----------------------------------------------------------------------------------------
\usepackage{amsmath,amsfonts,amssymb,amsthm}
\renewcommand{\qedsymbol}{$\blacksquare$}% Optional qed square





%----------------------------------------------------------------------------------------
%	PAGE HEADERS
%----------------------------------------------------------------------------------------

\usepackage{fancyhdr} % Required for header and footer configuration

\pagestyle{fancy}
\renewcommand{\chaptermark}[1]{\markboth{\sffamily\normalsize\bfseries\chaptername\ \thechapter.\ #1}{}} % Chapter text font settings
\renewcommand{\sectionmark}[1]{\markright{\sffamily\normalsize\thesection\hspace{5pt}#1}{}} % Section text font settings
\fancyhf{} \fancyhead[LE,RO]{\sffamily\normalsize\thepage} % Font setting for the page number in the header
\fancyhead[LO]{\rightmark} % Print the nearest section name on the left side of odd pages
\fancyhead[RE]{\leftmark} % Print the current chapter name on the right side of even pages
\renewcommand{\headrulewidth}{0.5pt} % Width of the rule under the header
\addtolength{\headheight}{2.5pt} % Increase the spacing around the header slightly
\renewcommand{\footrulewidth}{0pt} % Removes the rule in the footer
\fancypagestyle{plain}{\fancyhead{}\renewcommand{\headrulewidth}{0pt}} % Style for when a plain pagestyle is specified

% Removes the header from odd empty pages at the end of chapters
\makeatletter
\renewcommand{\cleardoublepage}{
\clearpage\ifodd\c@page\else
\hbox{}
\vspace*{\fill}
\thispagestyle{empty}
\newpage
\fi}


%----------------------------------------------------------------------------------------
%	ENVIRONMENTS
%----------------------------------------------------------------------------------------
\RequirePackage[framemethod=default]{mdframed}

\newcounter{dummy} 
\newtheorem{theoremeT}[dummy]{Stelling}
\newtheorem{problem}{Probleem}[chapter]
\newtheorem{exerciseT}{Oefening}[chapter]
\newtheorem{exampleT}{Voorbeeld}[chapter]
\newtheorem{definitionT}{Definitie}[section]



% Theorem box
\newmdenv[skipabove=7pt,
skipbelow=7pt,
backgroundcolor=black!5,
linecolor=hogentblue,
innerleftmargin=5pt,
innerrightmargin=5pt,
innertopmargin=5pt,
leftmargin=0cm,
rightmargin=0cm,
innerbottommargin=5pt,
skipbelow=2pt,
skipabove=2pt]{tBox}


\newmdenv[
skipabove=7pt,
skipbelow=7pt,
backgroundcolor=black!10,
rightline=false,
leftline=true,
topline=false,
bottomline=false,
linecolor=hogentblue,
innerleftmargin=5pt,
innerrightmargin=5pt,
innertopmargin=0pt,
leftmargin=0cm,
rightmargin=0cm,
linewidth=4pt,
innerbottommargin=0pt
]{definitionstyle}

\newmdenv[%
skipabove=7pt,
skipbelow=7pt,
rightline=false,
leftline=true,
topline=false,
bottomline=false,
linecolor=gray,
backgroundcolor=black!10,
innerleftmargin=5pt,
innerrightmargin=5pt,
innertopmargin=5pt,
leftmargin=0cm,
rightmargin=0cm,
linewidth=4pt,
innerbottommargin=5pt
]{examplestyle}

\newmdenv[%
skipabove=7pt,
skipbelow=7pt,
rightline=true,
leftline=true,
topline=true,
bottomline=true,
linecolor=gray,
backgroundcolor=black!10,
innerleftmargin=5pt,
innerrightmargin=5pt,
innertopmargin=5pt,
leftmargin=0cm,
rightmargin=0cm,
linewidth=1pt,
innerbottommargin=5pt
]{excersisestyle}


\newenvironment{definition}{\begin{definitionstyle}\begin{definitionT}}{\hfill{\tiny\ensuremath{\blacksquare}}\end{definitionT}\end{definitionstyle}}	
\newenvironment{example}{\begin{examplestyle}\begin{exampleT}}{\hfill{\tiny\ensuremath{\blacksquare}}\end{exampleT}\end{examplestyle}}	
\newenvironment{exercise}{\begin{excersisestyle}\begin{exerciseT}}{\hfill{}\end{exerciseT}\end{excersisestyle}}	
\newenvironment{theorem}{\begin{definitionstyle}\begin{theoremeT}}{\hfill{\tiny\ensuremath{\blacksquare}}\end{theoremeT}\end{definitionstyle}}	


% ---- MATH STUFF ----

\pgfmathdeclarefunction{gauss}{2}{%
  \pgfmathparse{1/(#2*sqrt(2*pi))*exp(-((x-#1)^2)/(2*#2^2))}%
}

\pgfplotsset{compat=1.11} % Insert the commands.tex file which contains the majority of the structure behind the template
\begin{document}
%\linenumbers

%----------------------------------------------------------------------------------------
%	TITLE PAGE
%----------------------------------------------------------------------------------------


\begingroup
\thispagestyle{empty}
\begin{tikzpicture}[remember picture,overlay]
\coordinate [below=12cm] (midpoint) at (current page.north);
\node at (current page.north west)
{\begin{tikzpicture}[remember picture,overlay]
\draw[anchor=north] (midpoint) node [fill=hogentblue!30!white,fill opacity=0.6,text opacity=1,inner sep=1cm]{\Huge\centering\bfseries\sffamily\parbox[c][][t]{\paperwidth}{\centering Analyse 1\\[15pt] % Book title
{\Large Hoe effici\"ent kwaliteitsvolle software opleveren}\\[20pt] % Subtitle
{\huge Dr. Jens Buysse}}}; % Author name
\end{tikzpicture}};
\end{tikzpicture}
\vfill
\endgroup

%----------------------------------------------------------------------------------------
%----------------------------------------------------------------------------------------
%	COPYRIGHT PAGE
%----------------------------------------------------------------------------------------


\newpage
~\vfill
\thispagestyle{empty}

\noindent Copyright \copyright\ 2015 Jens Buysse\\ % Copyright notice

\noindent \textsc{www.hogent.be}\\ % URL

\noindent \textit{Eerste uitgave, September 2014} % Printing/edition date

%----------------------------------------------------------------------------------------
%	TABLE OF CONTENTS
%----------------------------------------------------------------------------------------


\pagestyle{empty} % No headers

\tableofcontents % Print the table of contents itself

\cleardoublepage % Forces the first chapter to start on an odd page so it's on the right


 

% Commando voor invoegen Java-broncodebestanden (dank aan Niels Corneille)
% Gebruik: \codefragment{source/MijnKlasse.java}{Uitleg bij de code}
\newcommand{\codefragment}[2]{ \lstset{%
  language=java,
  breaklines=true,
  float=th,
  caption={#2},
  basicstyle=\scriptsize,
  frame=single,
  extendedchars=\true
}
\lstinputlisting{#1}}


\def\R{\mathbb{R}}


%VOEG HIER UW LESSEN TOE
\chapter{Introductie}

\section{Algemeen}
Dit zijn de cursusnota's bij het vak Analyse 1. In dit document ga je wat extra uitleg vinden die de slides moeten verduidelijken. 

De bedoeling is dat je de stof zelfstandig tot je zelf neemt en de oefeningen maakt per hoofdstuk. De oplossingen van uw oefeningen post je op het forum van het vak en worden besproken door alle studenten (en natuurlijk door de docent). 

Voor de oefeningen zal gebruik gemaakt worden van Visual paradigm, wat je kan downloaden op 
%
\url{http://www.visual-paradigm.com/download/}

Doorheen de cursusnota's vind je enkele opdrachten. De bedoeling is de opdrachten te bespreken op het forum. 

\section{Analyse}
Het vak analyse bereidt de studenten voor tot informatici die niet enkel en alleen verschillende soorten programmeertalen machtig zijn, maar ook weten welke methodieken gebruikt kunnen worden om een ICT project succesvol af te leveren. Factoren die het al dan niet falen van een project be\"invloeden zijn onder andere:

\begin{itemize}
	\item Tijdsbestek
	\item Financieel kader
	\item Manpower
	\item \dots
\end{itemize}

\begin{exercise}
	Welke eigenschappen en factoren kunnen nog een invloed hebben op het al dan niet falen van een ICT project? Welke zijn minder relevant?
\end{exercise}

De realiteit echter, toont aan dat maar al te vaak ICT projecten falen. Voorbeelden vind je wel overal zoals bijvoorbeeld het annuleren van een ERP (Enterprise Resource planning) project bij de luchtmacht van de verenigde staten nadat al 1 biljoen dollar gespendeerd was \cite{Kanaracus2012} \cite{failedProjects}.

\begin{exercise}
	Kan je enkele Belgische of Vlaamse ICT projecten vinden die onlangs gefaald zijn?
\end{exercise}

De auteurs van \cite{International1995} vergelijken het maken van software  met het bouwen van een brug. Ze vermelden dat bruggen vaak succesvol gebouwd worden doordat er veel details aanwezig zijn en dat het volledig beschreven is volgens bepaalde specificaties. Indien men dezelfde methodologie zou gebruiken voor het maken van software zou men wel eens bedrogen kunnen uitkomen. Het rapport toont ook aan dat maar 26\% van gestarte ICT projecten succesvol en binnen het tijdbestek en financieel plan afgeleverd werden. De rest van de projecten werden ofwel niet afgeleverd ofwel voldeden ze niet aan de eisen van de opdrachtgever. Nu moet dit rapport wel met de nodige kritische geest gelezen worden. Zier hiervoor bijvoorbeeld \cite{Glass2006}. Niettemin, is het wel een indicatie dat een softwareproject op een degelijke manier moet aangepakt worden. 

Het is in de lessen analyse dat we hiervoor de fundamenten zullen aanbrengen: kwaliteitsvolle ICT projecten opleveren.




\chapter{Het softwareontwikkelingsproces}
\section{De software crisis}

In het boek, ``The Mythical Man-Month'' \cite{Brooks1975} beschrijft Frederick Brooks een wet die later bekend werd als Brooks' Law:

\begin{center}
``Adding manpower to a late software project makes it later''
\end{center}

Brooks zegt zelf dat dit een simplificatie van het probleem is, maar wijst toch op twee erg belangrijke zaken:

\begin{enumerate}
	\item Het duurt even alvorens mensen die nieuw in een project toekomen productief worden. Software projecten zijn complexe ondernemingen en nieuwe projectmedewerkers moeten zich eerst een lange tijd inwerken alvorens ze mee zijn met het project zonder dat ze gedurende die periode een toegevoegde waarde kunnen leveren. Ook moeten de bestaande werkers tijd investeren in de nieuwe aangenomen projectmedewerkers. Daarenboven kan een nieuwe medewerker (daarom niet moedwillig) bugs introduceren in de code, waar 
	\begin{inparaenum}[(i)] 
		\item lang naar toe kan gezocht worden en 
		\item lang aan kan gewerkt worden om ze op te lossen.
	\end{inparaenum}
	\item Communiceren wordt erg moeilijk: het aantal mogelijke communicatiemogelijkheden stijgt polynomiaal bij het toevoegen van nieuwe medewerkers. 
\end{enumerate}

Blijkbaar helpt het toevoegen van meer mankracht in een project niet. De oorzaak van het falen van de projecten ligt dus blijkbaar ergens anders. 

\section{Oorzaken software crisis}
Er zijn twee soorten problemen die je kan tegen komen in de ontwikkeling van software.
\begin{itemize}
	\item Tamme problemen
	\item Gemene problemen (wicked)
\end{itemize}

\subsection{Tamme problemen}
Tamme problemen zijn stabiele en goed gedefinieerde problemen. Vaak zijn deze problemen goed afgebakend en is  er een pasklare oplossing.

Denk hierbij bijvoorbeeld aan:
\begin{itemize}
	\item Hardwareproblemen
	\item Ziekte van werknemer
	\item Tijdelijk uitvallen van netwerk
	\item \dots
\end{itemize}

\subsection{Wicked problems}
Wicked problemen daarentegen zijn vaag: je snapt pas het probleem eenmaal de oplossing ervoor geformuleerd is. Het is vaak onduidelijk wanneer ze opgelost zijn en ze zijn uniek voor de omgeving waarin je werkt. 

Denk hierbij aan:
\begin{itemize}
	\item Veranderende wens van de klant
	\item Introductie van nieuwe technologie\"en die je midden ontwikkeling zou willen gebruiken
	\item \dots
\end{itemize}

\begin{exercise}
	Vind zelf wat mogelijk tamme en wicked problemen.
\end{exercise}

\subsection{One wicked problem: clients}
Een groot wicked probleem is dat de opdrachtgevers vaak zelf niet goed weten wat ze willen. Of zoals Brooks het vermeldt:

\begin{center}
``\textit{For the truth is, the clients do not know what they want. They usually do not kwow what question must be answered, and they almost never have thought of the problem in the detail that must be specified.}''
\end{center}

en Jeff Atwood
\begin{center}
\textit{In software, we rarely have meaningful requirements.  Even if we do, the only measure of success that matters is whether our solution solves the customer's shifting idea of what their problem is.}
\end{center}

We moeten dus een manier vinden om met de wisselende wensen van de opdrachtgever om te gaan.

\begin{exercise}
	Lees het artikel van The Standish Group Report CHAOS en beantwoord voor jezelf de volgende vragen:
	
	\begin{enumerate}
		\item Waar gaat het artikel over?
		\item Voor wie is het artikel bedoeld?
		\item Hoe worden projecten volgens het artikel geclassificeerd?
		\item Wanneer slaag of faalt een project volgens het artikel?
		\item Hoe kan je de gestelde problemen oplossen?
		\item Welke conclusies trek je hier zelf uit en relateer dit met de stof gegeven in dit hoofdstuk.
	\end{enumerate}
\end{exercise}


\section{Softwareontwikkelingsproces}
Als beginnende ontwikkelaar van software ga je vaak de fout maken dat je onmiddellijk zou willen programmeren. Je kent makkelijk een aantal programmeertalen en leuke technische snufjes die je zou willen toepassen. Maar dit zou je gedurende het proces zuur kunnen komen te staan. 

Het is beter enkele vragen eerst te stellen, alvorens je de IDE erbij neemt.
\begin{itemize}
	\item Waar moet ik starten?
	\item Hoe moet ik er aan beginnen?
	\item Wat daarna?
\end{itemize}

Het antwoord op deze vragen is het volgen van een software ontwikkelproces. Een ontwikkelproces structureert je project zodat je bepaalde taken stap voor stap kan uitvoeren en je niet in 10 verschillende richtingen tegelijk werkt en alles in 1 keer klaar wilt hebben.

Een proces is een workflow die de volgorde vastlegt waarin je de dingen doet in een software project. 

Een analoog hierbij is volgende opdracht.

\begin{exercise}
Stel dat je een huis moet bouwen voor iemand. Het is hun droomhuis.  Ze hebben verschillende ide\"en op hun verlanglijst en je moet ook nog de normale dingen toevoegen die je in een huis kan vinden : muren, vloeren, keuken \dots

Vraag jezelf nu eens af waar je zou beginnen.
\end{exercise}

Formeel kan je het softwareontwikkelingsproces als volgt defini\"eren.
\begin{definition}
	Het software ontwikkelingsproces is  een kader dat vastlegt hoe een softwareproject wordt aangepakt en 
een methode om de activiteiten in verband met creatie, oplevering en onderhoud van softwaresystemen te organiseren.
\end{definition}

\subsection{Onderdelen van een SOOP}
Je kan verschillende onderdelen onderscheiden bij het het softwareontwikkelingsproces:
\begin{enumerate}
	\item De analyse
	\item Het ontwerp
	\item Implementatiefase
	\item Testfase
	\item Integratie
	\item Onderhoudsfase
\end{enumerate}

In wat volgt beschrijven we deze fasen kort:

\subsection{Analyse}
De studie en de beschrijving van het probleem: we moeten achterhalen wat het systeem moet doen en wat de randvoorwaarden zijn. We zorgen ervoor dat het probleem vertaald wordt in een taal die zowel verstaanbaar is voor de klant, als voor de mensen die het eigenlijke ontwerpen en implementeren voor zich zullen nemen. De analyse kan  verschillende componenten omvatten. We noemen er hier een aantal op en zullen verder doorheen de cursus verduidelijkt worden.
\begin{itemize}
	\item Use Cases (UC)
	\item User stories
	\item Opmaken van een domeinmodel
	\item Sequentiediagrammen
	\item \dots
\end{itemize}

Er moet gelet worden dat de software:
\begin{itemize}
	\item voldoet aan de taken en functies beschreven in de Use Cases en functionele specificaties;
	\item voldoet aan alle technische vereisten; 
	\item eenvoudig aan te passen is wanneer functionele eisen wijzigen.
\end{itemize}

\subsection{Ontwerp}
Deze fase omvat het specificeren van de nodige onderdelen van het te bouwen systeem: welke componenten zijn er nodig, kan ik hergebruiken en hoe werken die samen met elkaar. 
Het ontwerpmodel dient als abstractie van de broncode; het is een (gedetailleerd) plan dat beschrijft hoe de broncode wordt gestructureerd en geschreven. Het ontwerpmodel bestaat uit klassen, gestructureerd in modules en subsystemen, met duidelijk beschreven interfaces. Het bevat ook de beschrijving over hoe deze klassen met elkaar samenwerken. Je zal hier meer over leren tijdens de lessen Object ge\"orienteerd ontwerp.

\subsection{Implementeren}
Dit is het eigenlijke coderen van het systeem: we vertellen in een taal verstaanbaar voor de computer hoe die de verschillende functies moet uitvoeren.
Hierin komt naar voor:
 
 \begin{itemize}
		\item het vastleggen van de organisatie van de broncode;
		\item het implementeren van subsystemen;
		\item het vastleggen van klassen en objecten;
		\item het testen van de ontwikkelde componenten (Unittesten);
		\item het integreren van de resultaten van individuen tot een systeem.
\end{itemize}


\subsection{Testen \& Valideren}
In de analysefase worden een aantal testscenario's opgemaakt. In de testfase gaan we dus na of de software werkt en verwacht aan de verwachtingen van de klant. Hier komt naar voor:
\begin{itemize}
	\item Het controleren van de interactie tussen verschillende objecten;
	\item Het controleren van de integratie van verschillende componenten;
	\item Het controleren van het systeem om te kijken of alle requirements juist zijn ge\"mplementeerd;
	\item Het in kaart brengen en prioritiseren van defecten.
\end{itemize}

\subsection{Integratie}
Indien het systeem moet samenwerken met andere bestaande componenten of systemen, zal dat gebeuren in de integratiefase. Natuurlijk zal hier ook een test en validatiemoment moeten ingelast worden.

\subsection{Onderhoud}
Eenmaal de software af is, kunnen er nog wijzigingen doorgevoerd worden ten behoeve van veranderende eisen van de klant of vanuit een andere omgeving. 

\section{Watervalmethode}
Een eerste ontwikkelproces is de watervalmethode: bij de waterval methode, volgt de ene stap na de andere. De eerste stap moet volledig zijn afgerond alvorens aan de volgende kan worden gestart en er is geen weg terug. Elke fase voorziet cruciale activiteit(en), de fasen zelf zijn georganiseerd in een sequenti\"le en lineaire ordening waarbij elke fase volledig moet be\"indigd zijn voor de volgende kan aangevangen worden. De uitvoer van die fasen wordt gebruikt als invoer voor de volgende. Dit is het oudste ontwikkelingsproces : goed voor wel gekende systemen met weinig problemen en stabiele requirements. Maar dit is echter zelden het geval.

\begin{example}
	Het is augustus. Hartje zomer. In een grote vergaderruimte vindt de kick-off plaats van een groot project. Enthousiast presenteert de projectleider zijn planning. De analyse neemt 3 maanden in beslag. In november gaat het ontwerp van start. Medio januari van het volgend jaar leveren de ontwerpers op en aansluitend start de ontwikkeling van de nieuwe applicatie. Omdat de functionaliteit complex is neemt deze fase 6 maanden in beslag, zo is begroot. Dit betekent dat de applicatie, na uitvoering en testen in september wordt opgeleverd. Iedereen van opdrachtgevers tot ontwikkelaars, knikt instemmend. Een haalbare planning.
Inmiddels is men een jaar verder. Het is weer augustus en het ontwerp is nagenoeg gereed. ``We zijn op 90 procent '', zeggen de ontwerpers. Er spreekt meer hoop dan realisme. De projectleider extrapoleert de uitloop. In het huidige tempo wordt de applicatie in mei volgend jaar opgeleverd. Ruim een half jaar later dan gepland. En het project is nog niet halverwege. Nog meer uitloop is niet denkbeeldig.
\end{example}

Uit voorgaand voorbeeld kunnen we de nadelen van de watervalmethode opsommen:
\begin{enumerate}
	\item \textbf{Geen flexibiliteit}: omdat de ontwikkelfasen zo definitief zijn, gaan analisten, ontwerpers en ontwikkelaars behoedzaam te werk. Er is geen weg terug en dus moet het op te leveren werk volledig correct zijn. Dit voorkomt  dat er nieuwe wensen en eisen ontstaan tijdens een project bv doordat de gebruikers dit eisen. 
	\item \textbf{Verlies van informatie}: zodra een ontwerper klaar is met zijn fase gaat de ontwikkelaar ermee aan de slag. In de meeste projecten is dit het moment waarop de ontwerper het project verlaat. En al is hij nog grondig te werk gegaan, nooit is alles compleet. De meeste kennis zit in het hoofd van de ontwerper en deze kennis zal op dat moment niet meer voorhanden zijn.
	\item \textbf{Testen als sluitpost} : pas als de applicatie door de ontwikkelaars is opgeleverd wordt  er getest. Probleem : naarmate een fout later in het project wordt ontdekt, nemen de kosten voor het oplossen van de out functioneel toe. Het is belangrijk fouten zo vroeg mogelijk te detecteren. 
\end{enumerate}

\section{Iteratief en incrementeel werken: Agile}

Bij iteratief benaderen is er ook een weg terug. Agile software ontwikkeling is een conceptueel raamwerk voor het uitvoeren van softwareontwikkelingsprojecten als alternatief voor traditionele starre praktijken. Het Engelse woord agile betekent: behendig, lenig. De omslag heeft plaats gevonden vanwege de vele tekortkomingen die zich voor doen in de waterval methode. Lange (ontwikkel) trajecten, onvolledige en verouderde documentatie als ook een vast te volgen plan hebben allemaal een negatieve invloed op het eindproduct.
Agile ontwikkeling richt zich voor als voornaamste taak op het leveren van een product waar de klant tevreden over is. Agile projecten trachten dit te bereiken door het ontwikkelproces te benaderen met een proactieve houding naar de klant toe, als ook een dynamische en ``open voor verandering'' mentaliteit.

\subsection{Agile}
In februari 2001 kwamen 17 prominenten op het terrein van Agile-ontwikkeling bijeen. De groep stelde het Agile Manifest \cite{Kent} op, waarnaar puristische Agile-ontwikkelaars nog altijd graag verwijzen. Enkele principes uit het Manifest: 
\begin{itemize}
	\item Klanttevredenheid door snelle levering van bruikbare software op continue basis.
	\item Regelmatig aanbod van nieuwe werkende software (eerder per week dan per maand).
	\item Wijziging van doelstellingen zijn welkom, zelfs laat in het proces.
	\item Nauwe samenwerking op dagelijkse basis tussen ontwikkelaars en hun belanghebbenden.
	\item Direct persoonlijk contact.
	\item Eenvoud.
	\item Zelf-organiserende teams.
	\item Voortdurende aanpassing aan veranderende omstandigheden.
\end{itemize}

Agile ontwikkelingsmethoden zijn al veel ouder dan het manifest. Zo kent Scrum - een bekende Agile methode met vaste samenstelling bestaande uit een projectleider, een projecteigenaar en een team - zijn wetenschappelijke 'coming out' in 1986. Extreme Programming en DSDM zijn van medio jaren negentig, toen deze nog bekend stonden als 'light-weight-methods'.

\subsubsection{Contra Waterval}
Alle varianten van Agile methodes zetten zich af tegen het 'waterval-ontwikkelmodel': een model dat bestaat uit vast omlijnde fases waarbij het hoogste niveau als eerste wordt uitgevoerd en daarna de lagere - zoals een waterval. Een model dat oorspronkelijk afkomstig is uit de bouwsector, waar aanpassingen naderhand hand prijzig, zo niet onmogelijk zijn.

Dit watervalmodel - dat pas naar een volgende fase gaat als de vorige is afgesloten- zou te bindend zijn, te log en te bureaucratisch (veel papierwerk!). In tegenstelling tot de Agile methodes, die rekening kunnen houden met veranderende doelstellingen en zich bevrijden van bureaucratie door de nadruk op persoonlijk contact.

Tegelijkertijd lijkt dat ook de zwakte van Agile ontwikkeling: als plannen telkens veranderen en communicatie niet in documenten vastligt, ligt de weg open naar willekeur en chaos. Daarom moet er een organisatie staan die past bij de Agile methode: een open, communicatieve cultuur waarin competente mensen het vertrouwen krijgen om beslissingen te nemen.

\section{(Rational)Unified Proces}
Een vorm van Agile ontwikkeling is het Rational Unified Proces \cite{VantEinde2014} (kortweg RUP). Deze methode is gebaseerd op onderstaande best practices:

\begin{enumerate}
	\item \textbf{Ontwikkel software iteratief}: ontwikkelen in korte overzichtelijke perioden genaamd iteraties. Binnen de iteraties hebben de klassieke activiteiten zoals: planning, analyse, ontwerp, testen en documentatie allemaal een plaats zij het gericht op de omvang van de iteratie zelf. Een Agile project team bestaat dan ook uit personen die elk zich gespecialiseerd zijn in een van deze vaardigheden.
	\item \textbf{Ontwikkel software incrementeel}: zorg dat je na elke iteratie een werkend product aflevert. Naast de risico's te verminderen (er kan een volledig systeem getest worden), wordt ook aan de klant dan een deel van het product geleverd die direct gebruikt kan worden. Die kan dan ook zijn opmerkingen onmiddellijk meegeven. Dit verkleint het risico dat het eindproduct niet is wat de klant wil en de klant kan nu per deelproduct zijn feedback geven wat je weer kunt meenemen tijdens de ontwikkeling van het volgende deelproduct. De fasen worden herhaaldelijk doorlopen. 
	
	\end{enumerate}
Daarnaast worden nog enkele raadgevingen meegegeven aan de mensen in het project:
	\begin{enumerate}
	\item \textbf{Maak gebruik van component gebaseerde architectuur}: Systemen met een op componenten gebaseerde architectuur zijn eenvoudig uit te breiden, inzichtelijk, begrijpelijk en bevorderen het hergebruik van bepaalde delen code. Aangezien de systemen steeds groter worden neemt het belang van een goede architectuur toe. RUP is erop gericht de basisarchitectuur in een vroeg stadium te bepalen, en naarmate het systeem groter wordt zal de architectuur zich verder uitbreiden. Bij iteratief ontwikkelen is het mogelijk de componenten geleidelijk in kaart te brengen om ze vervolgens te ontwikkelen, te kopen of te hergebruiken.
	\item \textbf{Maak gebruik van prototypes}: Door de gebruiker een grafische voorstelling te geven van het product (prototyping), verkleint de faalkans van het project. Een globale, grafische oplossing voor het probleem is door de gebruiker beter te begrijpen dan pagina's vol broncode. Het is een versimpeling van de complexiteit. Naast prototypes komen in deze fase ook use cases, use case diagrammen, klassendiagrammen en andere objecten naar voren.
	\item \textbf{Test het systeem}: Het bepalen van de kwaliteit van een systeem gebeurt op basis van testen. Dit is een van de punten waarop software projecten vaak falen omdat het testen vaak aan het einde van het project wordt gedaan, soms helemaal niet en soms door andere teams. RUP vangt dit probleem af door het testen in het gehele proces terug te laten komen en daarbij alle belanghebbenden (stakeholders) te betrekken (zowel programmeurs als klanten). RUP gaat er vanuit dat elke belanghebbende verantwoordelijk is voor de kwaliteit gedurende het gehele project
	\item \textbf{Maak gebruik van versiebeheer tijdens de software-ontwikkeling}: Zoals bij alle andere software-projecten zijn veranderingen in de software onvermijdelijk. RUP beschrijft een aantal methoden om deze veranderingen te beheersen en nauwkeurig te volgen. RUP beschrijft ook beveiligde werkruimtes, hierin staat bijvoorbeeld dat het systeem van een programmeur niet aangetast mag worden door veranderingen in een ander systeem.
\end{enumerate}


\section{Object Ge\"ori\"enteerde analyse}
In de lessen analyse gaan we object geori\"enteerd (OO) te werk. In dit stuk halen we kort aan wat deze term inhoudt.

OOP streeft er naar om een project zo structureel mogelijk op te bouwen in objecten en klassen. Dit heeft voor de programmeur het grote voordeel dat code vanaf nu in logische componenten wordt opgedeeld en veel makkelijker te hergebruiken is.

Om het concept van objecten te illustreren kan je bijvoorbeeld denken aan een auto Audi A4. De auto is het object en dit object heeft bepaalde eigenschappen. Een eigenschap van de auto kan bv de kleur, zetelbekleding \dots zijn. Een auto heeft ook een aantal onderdelen waaruit het bestaat, die ook voorgesteld kunnen worden als object. Denk hierbij maar aan een deur, een band \dots Maar een auto heeft ook functies. Een functie kan starten of remmen zijn. Dus hebben we nu eigenlijk een object met eigenschappen en functies die in relatie kan staan met andere bestaande objecten. Het is op deze manier dat we ons modellering gaan toepassen. 

\begin{definition}
	In een object georie\"nteerde methodologie worden zaken beschreven aan de hand van klassen. Klassen bevatten eigenschappen en methoden/functies. Een instantie van een dergelijke klasse is een object: het bevat de concrete waarden voor de eigenschappen van zijn klasse en heeft de mogelijkheid de methoden en functies uit te voeren.
\end{definition}



\section{UML}
De Unified Modelling Language (UML) is een taal om diagrammen te maken of een notatiewijze om modellen van objectgeori\"nteerde softwaresystemem te specificeren, te visualiseren en te documenteren \cite{Hensgen2003}. UML is geen ontwikkelmethode, d.w.z. het vertelt je niet wat je eerst moet doen en wat daarna, of hoe u uw systeem moet ontwerpen, maar het helpt je om uw systeem te visualiseren en te communiceren met anderen. UML staat onder toezicht van de Object Management Group (OMG) en is de industriestandaard voor het grafisch weergeven van software. 

UML is opgebouwd uit vele modelelementen die de verschillende delen van een softwaresysteem vertegenwoordigen. De UML-elementen worden gebruikt om diagrammen te maken, die een bepaald deel of een gezichtspunt van een systeem voorstellen. 

\subsection{Diagramma}
In wat volgt introduceren we de diagrammen die we zullen gebruiken doorheen de cursus.
\subsubsection{Use Case diagrammen}
Use case-diagrammen beschrijven de relaties en afhankelijkheden tussen een groep van use cases en de actoren die deelnemen aan het proces.

Belangrijk om op te merken is dat use case-diagrammen niet geschikt zijn om het ontwerp te representeren, en niet het inwendige van een systeem kunnen beschrijven. Use case-diagrammen zijn bedoeld om de communicatie met de toekomstige gebruikers van een systeem, en met de klant, te vergemakkelijken, en zijn in het bijzonder behulpzaam bij het vaststellen van welke benodigde kenmerken een systeem moet hebben. Use case diagrammen vertellen wat het systeem moet doen maar specificeren niet - en kunnen dat ook niet - hoe dit gerealiseerd moet worden.

\subsubsection{Klassendiagrammen}
Klassendiagrammen tonen de verschillende klassen waaruit het systeem is gemaakt, en hoe zij aan elkaar gerelateerd zijn. Van klassendiagrammen zegt men dat zij \textit{statische} diagrammen zijn omdat zij weliswaar de klassen weergeven, samen met hun methoden en attributen, alsmede de statische relaties tussen hen (i.e. welke klassen ''hebben weet`` van welke klassen of welke klassen ''maken deel uit`` van een andere klasse), maar niet de methode-aanroepen tussen hen onderling weergeven. 

\subsubsection{Activiteitsdiagrammen}
Activiteitsdiagrammen beschrijven de volgorde van activiteiten in een systeem net behulp van activiteiten. Activiteitsdiagrammen zijn een bijzondere vorm van toestandsdiagrammen, die alleen (of voornamelijk) activiteiten bevatten. 

\subsubsection{Sequentiediagrammen}
Deze geven de berichtenuitwisseling weer (bijv. methode-aanroep) tussen verscheidene objecten in een specifieke tijd-begrensde situatie. Objecten zijn instanties van klassen. Sequentiediagramma leggen speciale nadruk op de volgorde waarin en de tijdstippen waarop de berichten naar de objecten verstuurd worden.

In volgordediagrammen worden objecten gerepresenteerd door verticale onderbroken lijnen, met de naam van het object bovenaan. De tijd-as loopt ook verticaal, en neemt toe naar beneden, zodat berichten verstuurd worden van het ene object naar de nadere in de vorm van pijlen met de namen van de operatie en de parameters erbij. 

\subsection{Voordelen van UML}
\begin{itemize}
	\item UML wordt gebruikt in een verscheidenheid van doeleinden en de leesbaarheid en herbruikbaarheid maken het een ideale keuze voor ontwikkelaars.
	\item De visualisatie vergemakkelijkt de communicatie tussen de ontwikkelaars en de opdrachtgever.
	\item Door informatie in een diagram te verwerken, is het gemakkelijk om relaties van een programma te begrijpen en te visualiseren . 
	\item Visualisatie maakt het gemakkelijk voor een nieuwe programmeur om te stappen in een project en productief te zijn vanaf dag een. 
	\item  Een aantal UML tools kunnen code genereren op basis van de diagramma. Dit vermindert overhead.
	\item \dots
\end{itemize}

\begin{exercise}
	Zoek zelf een aantal extra voordelen van UML.
\end{exercise}


\chapter{Use Cases}

\section{Inleiding}

Om direct met de deur in huis te vallen geven we hier de definitie van een Use Case (of kortweg UC) zoals beschreven in \cite{Jacobson2012}.
\begin{definition}
	Een use case omvat alle manieren waarop het systeem gebruikt kan worden om een bepaald doel voor een bepaalde gebruiker te behalen. Een complete set use cases geeft je alle zinvolle manieren om het systeem te gebruiken en illustreert de waarde die dit zal opleveren.
\end{definition}

Daarbijhorend geven we ook onmiddelijk al de definitie van een scenario:

\begin{definition}
	Een scenario is een opeenvolging van stappen die beschrijven hoe een gebruiker of entiteit buiten het systeem interageert met dat systeem. 
\end{definition}

Een andere definitie van UC kan dan als volgt gegeven worden:

\begin{definition}
	Een use case is een set van scenario's die samen verwerkt zijn om een gedeelde user goal te bereiken.
\end{definition}

UC's worden vaak gebruikt om het verhaal dat de opdrachtgever vertelt aan de ontwikkelaars te structureren en begrijpbaar te maken. Use cases maken inzichtelijk wat een systeem moet doen en, door doelbewuste weglating, wat het
systeem niet zal doen. Ze maken ideevorming, scope management en incrementele ontwikkeling mogelijk van ieder type systeem van iedere omvang. Neem het voorbeeld hieronder: de gebruiker vertelt een verhaal maar het is nog niet duidelijk wat het systeem (niet) moet kunnen met wie/wat.

\begin{example}
Onze klant wil graag een simulatie krijgen van een dobbelspel.  De klant meldt ons meteen dat als dit spelletje aanslaat, er dan heel snel uitbreidingen kunnen volgen. Hij beschrijft ons hieronder wat dit simulatiespel allemaal moet kunnen: ''Het spel is vrij eenvoudig``. Een speler gooit met twee dobbelstenen. Die eerste worp is belangrijk. Werpt hij 7 of 11, dan wint de speler onmiddellijk het spel met een score 2. Anders moet hij dit specifieke aantal ogen - van de eerste worp - nog eens werpen vooraleer hij een 7 of een 11 werpt, ook dan wint hij maar met een score 1. Een speler kan pas spelen als hij voldoende krediet heeft. Een nieuwe speler start met een krediet van 5, hij kan een aanvraag indienen om dat krediet te verhogen - een beheerder controleert de betalingen. Het krediet kan verhoogd worden door te betalen of door spelletjes te winnen. Het is een spel dat iedereen dus afzonderlijk kan spelen, maar we willen als dit aanslaat in een verder stadium ook graag tornooien opstellen.  Elke speler moet  gekend zijn in het spel zodat we een klassement kunnen opstellen. We organiseren dit nu als volgt: afhankelijk van het aantal deelnemers, starten we met een aantal tafels. Stel dat er vijf mensen aan 1 tafel zitten, dan mag iedere speler dit spel $5 + 1$ keer spelen. Van die tafel wordt een rangorde opgesteld, de twee beste gaan door naar de volgende rond. Daar worden terug groepjes gemaakt. Terug de twee beste gaan door.  Dan kom je aan de ronde waar er maar drie tafels meer overschieten.  Hier gaat slechts telkens de beste door. Deze drie beste spelers spelen dan de finale. Soms worden de spelers willekeurig in groepen verdeeld om het tornooi te starten, soms worden ze op basis van vorige klassementen bij elkaar geplaatst. Daarnaast heb ik ook nog allerhande extra mogelijkheden in mijn hoofd: er bestaan ook spelvarianten van dit dobbelspel. Een eerste voorbeeld hiervan is dat de speler maximaal vijf maal mag gooien met de dobbelstenen.  Een ander voorbeeld kan zijn dat het cijfer 2 ook zorgt - naast een 7 of een 11 - dat de speler wint.  Nog een ander wijziging die mogelijk is, is het aantal wedstrijden dat per ronde gespeeld wordt. De mogelijkheden zijn eindeloos, maar misschien moeten we eerst zorgen dat we het basisspel via pc kunnen spelen?
\end{example}

\subsection{Basisprincipes}
Er zijn zes basisprincipes die ten grondslag liggen aan iedere succesvolle toepassing vna use cases:

\begin{enumerate}
	\item Simpliciteit
	\item Het grotere geheel begrijpen
	\item Focus op waarde
	\item Bouw het systeem op in slices
	\item Lever het systeem incrementeel op (zie verder)
	\item Sluit aan op de behoeften van het team
\end{enumerate}

\section{Onderdelen van een UC}

Een UC is opgedeeld uit verschillende onderdelen. 
\begin{enumerate}
	\item We beginnen met de beschrijving van het \textbf{main succes scenario (MSS)} of \textbf{normaal verloop}. Je beschrijft dit als een genummerde sequentie van stappen. Deze stappen moeten eenvoudig zijn en beschrijven wie welke stap uitvoert. 
	\item Andere scenario's beschrijf je als variaties van het MSS: \textbf{extenties} of \textbf{alternatief verloop}. Deze kunnen succesvol aflopen of fouten zijn die het systeem moet aankunnen. Een extentie start bij het nummer waar het in het MSS anders loopt en beschrijft kort de reden waarom het anders verloopt. De volgende stappen worden opnieuw genummerd dan (zoals in het MSS). 
	\item Elke use case heeft een \textbf{primary actor} of \textbf{primaire actor}, die het systeem gebruikt voor een bepaalde service. Er kunnen meerdere primaire actoren zijn. Let op, een actor hoeft niet per definitie een mens te zijn, het kan ook een ander softwarepakket of entiteit zijn.
	\item Een korte beschrijving legt kort uit waarvoor de UC dient.
	\item \textbf{Precondities} beschrijven de voorwaarden waaraan moeten voldaan zijn alvorens de UC succesvol kan aflopen.
	\item \textbf{Postconditites} beschrijven de voorwaarden (de staat waarin het systeem zich bevindt) waaraan voldaan moeten zijn als de UC succesvol afloopt. 
	\item \textbf{Domeinregels} zijn  regels die opgelegd worden aan onderdelen van de UC. 
\end{enumerate}

\section{UC diagramma}
Het use case diagram \cite{UCDiagram} is een grafische weergave van de scope van het systeem waar duidelijk wordt wat de actoren zijn en wat hun relaties zijn met de use cases.
 
Het use case diagram bestaat uit stokfiguren, ellipsen, lijnen en kaders. Daarnaast wordt er ook nog gebruik gemaakt van de pijlrelaties include en extend.

\subsection{Use Cases}
\begin{itemize}
	\item Use cases zijn ellipsen in het diagram en bevatten een korte maar krachtige omschrijving.
	\item De omschrijving is een combinatie van een werkwoord en een zelfstandig naamwoord dat wordt gebruikt in het domein van de belanghebbenden.
\end{itemize}

\subsection{Actoren}
\begin{itemize}
	\item Actoren zijn personen, organisaties of externe systemen en worden voorgesteld door een stokfiguur.
	\item Actoren staan buiten het systeem maar communiceren wel direct met het systeem.
	\item Een entiteit kan meerdere actoren representeren omdat deze meerdere rollen kan hebben.
	\item Een actor die belang heeft om een bepaald doel te halen is een primaire actor.
\end{itemize}

\subsection{Relaties}
\begin{itemize}
	\item  Een relatie bestaat zodra de actor betrokken is bij een use case. De primaire actor zet vaak zelfs de use case in gang.
  \item Relaties tussen use cases en actoren worden als vaste lijnen weergegeven.
\end{itemize}

Er zijn twee speciale relaties tussen UC's:
\begin{itemize}
	\item Includes: een verplichte uitvoering van een deel Use Case in een andere use case.
	\item Extends: een afwijking van het normale verloop
\end{itemize}

Het is good practice om een Use Case diagram niet overdreven ingewikkeld te maken. Vermijd dus in de mate van het mogelijke includes en extends

\subsection{Het systeem}
\begin{itemize}
	\item Het systeem wordt voorgesteld door een rechthoek en duidt de scoop aan van het systeem. 
	\item Kaders worden alleen om use cases heen gezet. Alle use cases buiten het kader behoren niet tot de scope.
\end{itemize}

\section{Belang van Use Cases}

\subsection{Specificatie van functionele eisen}
Zoals al beschreven hierboven gaan we in een UC modelleren wat het systeem moet kunnen in interactie met een actor - dit heet men functionele vereisten.

\subsection{Eenheid van planning}
Use cases kunnen ook gebruikt worden als eenheid van planning om uw software af te leveren. Dit kan kort in volgende stappen:
\begin{enumerate}
	\item Identificeer de benodigde use cases.
	\item Sorteer de use cases volgens prioriteit
	\item Schat de ontwikkeltijd van de use cases
	\item Bepaal de lengte van uw iteratie en lever de ge\"implementeerde UC's incrementeel af.
\end{enumerate}

Een use-case hoeft dan geen complete beschrijving te zijn en al zijn testgevallen te bevatten. Je kan beginnen met de basis use case en slechts een testgeval. Met aanvullende UC's kun je dan het systeem compleet maken en veiligstellen dat alle testgevallen worden meegenomen. Dit is een zeer
flexibel opdeelmechanisme, dat je in staat stelt om iteraties te maken die precies groot of klein genoeg zijn
om de ontwikkeling goed te kunnen aansturen.


\subsection{Basis voor aanmaken functionele testen}
\textit{Use case testing} is een techniek die ons in staat stelt om test cases te ontdekken die het systeem testen van begin tot einde. Elk verschillend scenario moet uitgewerkt worden in de software en moet dus ook getest worden. Onze software zal voldoen aan de functionaliteitseisen enkel en alleen als elk scenario inderdaad werkt en blijft werken. Dit moet grondig uitgetest worden.


\subsection{Basis voor analyse en ontwerp}
De analyse in functie van UC's wordt later gebruikt bij het ontwerp (zie de lessen Ontwerp).

\subsection{Basis voor UI storyboard}
Zie slides

\section{Activity Diagram}
Een activiteitendiagram is een techniek die toelaat om procedurele logica en workflow te visualiseren. Use Cases scenario zijn procedureel en kunnen dus visueel voorgesteld worden via een activiteitendiagram.
In wat volgt beschrijven we de methode en componenten van een AC-diagram.

\begin{enumerate}
	\item Start met een initial node, voorgesteld door een zwarte cirkel
	\item Maak voor elke stap in uw scenario een actie (action), voorgesteld door een rechthoek.
	\item Verbind elke action met een lijnstuk met pijl, die de workflow aanduidt.
	\item Voor elk alternatief scenario ga je naar de actie waar het alternatief scenario begint. 
	\begin{enumerate}
		\item Teken  een decision node (keuzeknoop) voorgesteld door een ruit. Deze actie waar het alternatief scenario begint verbindt je met de decision node.
		\item Zo'n decision node heeft een guard, een voorwaarde die aanduidt welke pijl gevolgd moet worden. Deze guard schrijf je tussen vierkante haakjes (bv. $[x=1]$). 
		\item Bij elke pijl die vertrekt uit de decision node schrijf je de invulling van die guard (ook tussen vierkante haakjes).
	\end{enumerate}
	\item Bij het einde van de worklow maak je een final activity node, voorgesteld door een zwart cirkel omringt door een lege cirkel. Je kan hierbij nog een beschrijving zetten van de staat waarin je ge\"eindigd bent.
\end{enumerate}

\chapter{Klasse diagram - Domeinmodel}

\section{Introductie}
Vaak als je over UML diagramma spreekt, wordt gesproken over klasse diagramma. 

\begin{definition}
Een klassendiagram (Engels: class diagram) is een formele representatie van concepten en hun onderlinge statische relaties.
\end{definition}

Klassediagramma tonen ook de eigenschappen en operaties van een bepaalde klasse en 
en de beperkingen op de relaties tussen de klassen. 

Het domeinmodel is daarbij een speciale versie van het klassendiagram,
waarbij technologie specifieke zaken nog weg gelaten worden. Ze worden vooral
gebruikt om de statische relaties tussen de klassen te tonen en moeten 
verstaanbaar zijn voor zowel de opdrachtgever van het softwareproduct
als de ontwerpers en de mensen verantwoordelijk voor de implementatie. 

\subsection{Nut van domeinmodel}

\begin{itemize}
\item Domeinklassen zijn heel belangrijk omdat je daarmee de kern van het probleem (en de oplossing) met behulp van een betrekkelijk klein aantal begrippen kunt beschrijven.

\item Met behulp van het domeinmodel kan je nog eens afstemmen met de klant of je heel goed begrijpt hoe alles ineen zit (relaties tussen klassen).

\item Conceptuele domeinklassen vormen dan ook een belangrijk uitgangspunt en inspiratiebron voor het ontwikkelen van de softwareklassen ( klassen met datatypes en gedrag) waaruit een domeinlaag in een applicatie is opgebouwd.
\end{itemize}

\section{Klassendiagram en UML}
Het domeinmodel is een  visuele representatie van concepten uit de werkelijkheid en hun onderlinge relatie. We visualiseren dit via een UML klassendiagram. 

\subsection{Klassen}
Een klasse is het basisblok van een domeinmodel. Het wordt voorgesteld door een rechthoek
met bovenaan de naam van de klasse. 

\subsection{Properties of eigenschappen}
Een property (eigenschap) beschrijft een structurele eigenschap van de klasse. Deze 
komen in twee vormen voor: 
\begin{enumerate}
	\item Attribuut
	\item Associatie
\end{enumerate}

\subsubsection{Attribuut}
Een attribuut beschrijft een eigenschap als een stuk in de klasse rechthoek. In 
de analyse gaan we nog geen type, multipliciteit en zichtbaarheid meegeven bij
een attribuut. Dit komt wel aan bod bij ontwerpen.

\subsubsection{Associatie}
Een andere manier om een eigenschap te modelleren is via een associatie. Een associatie
is een rechte lijn tussen twee klassen, en kan gericht zijn met een bron en eindklasse. We gebruiken een associatie indien we vaststellen  dat er, gedurende het bestaan (of een deel er van) van de deelnemende klassen een relevant verband of een zinvolle relatie bestaat.

Deze richting is niet verplicht en duidt erop dat de associatie in twee richtingen kan gelezen
worden.  Een associatie heeft altijd een volgende eigenschappen:
\begin{itemize}
	\item De naam van de associatie of associatienaam
	\item Een multipliciteit: via een multipliciteit kunnen we aangeven hoeveel instanties van de ene klasse verbonden zijn met 1 instantie van de andere klasse. Let op: De multipliciteit communiceert hoeveel instanties geassocieerd kunnen worden met een andere instantie, op een bepaald moment en niet over een bepaald tijdsbestek. Er wordt vaak een minimum- en maximummultipliciteit gedefinieerd.
	
\begin{itemize}
	\item Minimum
			\begin{itemize}
			\item Optioneel : 0
			\item Verplicht (mandatory) : 1
		\end{itemize}
	\item Maximum
	\begin{itemize}
		\item Met 1 enkele waarde : 1
		\item Met meerdere waarden (multivalued) : n of *
	\end{itemize}
\end{itemize}

\end{itemize}

Het kan optioneel ook volgende eigenschappen bevatten:
\begin{itemize}
	\item Een rolnaam bij het einde van de associatie
	\item De leesrichting van de associatie.
\end{itemize}

\begin{exercise}
	Een reflexieve associatie is een associatie van een klasse met zichzelf. Kan dergelijke
	associatie bestaat met minimummultipliciteit 1? 
\end{exercise}

\subsubsection{Wanneer welke modellering}
Zowel attributen als associaties kunnen ongeveer dezelfde informatie overbrengen.
Je gebruikt een attribuut voor kleine eigenschappen, zoals bijvoorbeeld de kleur van 
een auto. Associaties gebruik je voor belangrijke eigenschappen, vaak tussen meerdere
klassen.

\begin{exercise}
	Maak de oefeningen vanuit de slides (schaakspel) en de andere oefeningen. De oefeningen kan je bespreken op het forum.
\end{exercise}

\section{Speciale associaties}
\subsection{Generalisatie}
Wanneer twee klassen veel eigenschappen gemeenschappelijk hebben, kunnen we dat modelleren aan de hand van generalisatie. In UML, plaatst een generalisatie-associatie tussen twee klassen hen in een hi\"erarchie die het concept van overerving van een afgeleide klasse van een basisklasse representeert. Zo is bijvoorbeeld de klasse Gebruiker een generalisatie van de klasse student, of met andere wooredn, een student is een bepaald type gebruiker. In UML, worden generalizaties weergegeven door een lijn, die de twee klassen met elkaar verbindt, met een pijlpunt aan de kant van de basisklasse. 

\subsection{Associatieklasse}
Een associatieklasse laat je toe om attributen en andere eigenschappen
toe te voegen aan een associatie. Een associatieklasse voegt een extra beperking toe, 
zodat er maar een instantie (object) van de associatieklasse kan bestaan tussen de 
twee participerende objecten.  Een associatieklasse is 

\begin{itemize}
	\item gerelateerd aan een associatie
	\item heeft een levensduur  afhankelijk van de associatie tussen de participerende objecten.
\end{itemize}

\subsection{Aggregatie}
Aggregatie duidt op een deel-van relatie tussen twee klassen. Bijvoorbeeld waarbij de 
klanten gezien worden gezien als onderdeel van een bank. Of waarbij werkmateriaal zoals
een bijl en een zaag gezien worden als onderdeel van een gereedschapskist. Hierbij is het 
dus belangrijk dat de relatie een relatie is tussen gelijken: een object van de ene klasse
kan bestaan zonder het object van de andere klassen (een bijl kan bestaan zonder 
gereedschapskist).

Een dergelijke aggregatie wordt aangegeven bij een open ruit, waarbij de  open ruit het 
geheel voorstelt (de gereedschapskist) en het andere deel is het deel (de bijl).

\subsection{Compositie}
Een compositie is opnieuw een relatie tussen twee klassen waarbij de focus ligt op de 
deel-geheel structuur. Enkel geldt de voorwaarde dat de ene klasse zonder de andere kan
niet: een van de deelnemende klassen heeft de andere nodig om te bestaan. Denk hierbij
aan het voorbeeld van het schaakbord dat bestaat uit verschillende vakjes. Een vakje
kan niet bestaan zonder schaakbord.

Een dergelijke compositie wordt voorgesteld door een volle ruit aan de klasse die het geheel
voorstelt. De deelklasse wordt verbonden via een lijn.

Bij compositie geldt dat:
\begin{itemize}
	\item een instantie van het onderdeel behoort tot juist 1 instantie van het geheel
	\item een instantie van een onderdeel behoort steeds toe tot een instantie van het geheel, het bestaat dus nooit alleen
	\item een instantie van het geheel is verantwoordelijk voor de creatie en vernietiging van zijn onderdelen
\end{itemize}

\subsubsection{Klein woordje over compositie en aggregatie}
Compositie en aggregatie worden enkele gebruik voor deel-geheel relaties te duiden. Het zijn
dan op zich dus ook gewoon relatie. Indien je dus twijfelt of iets een aggregatie
of compositie is, kan je het best een gewone associatie gebruiken want een aggregatie/compositie kan altijd voorgesteld worden door een associatie. Omgekeerd natuurlijk niet!

\begin{exercise}
	Leg in je eigen woorden vorige statement uit. Waarom kan een aggregatie of compositie altijd voorgesteld worden door een associatie, maar een willekeurige associatie niet door een compositie of aggregatie?
\end{exercise}

\subsection{Afhankelijk}
Een afhankelijkheid (dependency) bestaat tussen twee elementen indien een verandering van het ene element een verandering van het andere element meebrengt. 

Een voorbeeld maakt het misschien duidelijk. Een scherm wordt gemaakt die de informatie
van een klant voorstelt in een bank. Dan kan er een afhankelijkheid bestaan tussen de klasse
scherm en klant, want indien er iets verandert aan de informatie van de klant (we willen
bijvoorbeeld zijn geboortedatum ook bijhouden), dan zal het scherm moeten aangepast worden
want we willen dit ook tonen. 

Een afhankelijk bestaat maar in 1 richting en de richting is van de afhankelijke klasse naar
de onafhankelijke klasse.

Er is een gevaar bij gebruik van afhankelijkheden: als je veel afhankelijkheden gebruikt in een complex systeem, dan zal een kleine veranering in het systeem wel eens een cascade van
veranderingen meebrengen in het systeem. Opgepast dus!

\begin{exercise}
	Leg in uw eigen woorden uit waarom het gebruik van afhankelijkheden af te raden is. Gebruik hiervoor een voorbeeld.
\end{exercise}






\chapter{Systeem Sequentie Diagram}

\section{Sequentie diagram}

\begin{definition}
Een sequentiediagram is een diagram dat hoofdzakelijk toont hoe een sequentie van interacties tussen verschillende objecten verloopt, in de volgorde dat de interacties zich voordoen. 
\end{definition}

Een sequentie diagram beschrijft hoe groepen objecten met elkaar communiceren in het verwezenlijken van het systeemgedrag. Deze samenwerking wordt uitgevoerd als reeks berichten tussen objecten. Het SD beschrijft de gedetailleerde implementatie van een enkele use case (of een variatie van een enkele use case). De sequence diagrammen zijn niet nuttig om het gedrag binnen een object te tonen. Overweeg om voor dat doel een state-transition diagram te gebruiken.

Ze worden gebruikt om aan te geven hoe verschillende objecten met elkaar omgaan. Een van de belangrijkste aspecten van een SD is dus om de functionele vereisten van een Use Case te gaan verfijnen. Ook zijn ze vaak een erg goede visueel hulpmiddel om deze functionele vereisten te modelleren.

\section{Voorstelling}

Een SD wordt dus gebruikt om sequenties van boodschappen te modelleren. De nadruk ligt niet zozeer op de boodschappen zelf, maar wel op de volgorde waarin deze boodschappen verzonden worden. Niettegenstaande, gaat een SD wel defini\"eren welke boodschap er verzonden wordt tussen objecten. Een SD vertaalt deze informatie in een horizontale en verticale dimensie: 
\begin{itemize}
	\item Verticaal: de tijd waar de messages gebeuren 
	\item Horizontaal: de objecten die de messages versturen en ontvangen.
\end{itemize}

\subsection{Lifeline}
Lifelines stellen een rol of een object voor die participeren in de sequentie die gemodelleerd wordt. Ze worden getekend als een rechthoek met een stippellijn die recht naar beneden gericht is. Deze levenslijn stelt de tijd voor : deze loopt van boven naar beneden. Indien het object niet wordt vernietigd tijdens de collaboratie zal deze levenslijn dus tot onderaan het diagram doorlopen. De naam van de rol of het object wordt in de rechthoek geplaatst. Deelnemers worden als volgt weergegeven ``\texttt{Naam: Type}''  Meestal wordt alleen het type vermeld voorafgegaan door een :. Hieraan kan je zien dat het om een object gaat. Voor een specifiek object is in elk geval het benoemen van het type, de klasse verplicht. De naam van het object weerspiegelt de rol die dat specifiek object als participant heeft in het diagram en is optioneel. Men spreekt van een anoniem object. (object zonder naam of waarvan de naam niet relevant is). We kunnen dit weg laten als de rol van de participant direct duidelijk is, of als dat object het enige is van het bewuste type in het diagram. Bvb in 4 op een rij zijn er 2 spelers, dan moet je wel een onderscheid maken tussen speler1 en speler2. (In eerdere versies van UML werden objecten onderstreept, in UML 2.0 echter niet meer)



\subsection{Messages of berichten}

\subsubsection{Synchrone berichten}
\begin{definition}
Een bericht is een communicatie tussen objecten die informatie overbrengt in de verwachting dat er een actie wordt ondernomen
\end{definition}
Als een bericht wordt ontvangen start een activiteit in het ontvangende object : dit noemt men activering. Als een object een boodschap ontvangt is het actief, en blijft actief tot het de boodschap volledig heeft afgehandeld. De activering toont de focus van de besturing: welke objecten op een bepaald moment worden uitgevoerd. Een geactiveerd object voert ofwel zijn eigen code uit of wacht op de return van een ander object waaraan het een bericht heeft gestuurd. Activering wordt getekend als een smalle rechthoek op de levenslijn. 

Het eerste bericht wordt bovenaan getekend, meestal helemaal links met de opeenvolgende messages eronder. 

Een message wordt voorgesteld door een volle lijn met een volle pijl. De naam van de message wordt bovenaan de pijl gezet met eventuele parameters. Wat een dergelijke message symboliseert is het eigenlijke aanroepen van een bepaalde methode van het object dat aan de pijlzijde van de message staat.

\subsubsection{Return messages of terugkeerwaarden}
Naast asynchrone messages kunnen ook waarden terug gegeven worden. Dit wordt voorgesteld door een pijl met stippellijn en een niet gevulde pijlpunt. Boven de stippellijn plaats je de naam van de waarde die je teruggeeft. 


\subsection{Loop}
Het is mogelijk om een loop te defini\"eren: je tekent hiervoor een rechthoek rond de messages die je wil herhalen en je schrijft de loopvoorwaarde linksbovenaan in de rechthoek. 

\subsection{Alternatief}
Het is mogelijk om gebaseerd op een bepaalde waarde een sequentie uit te voeren. Je tekent hiervoor een rechthoek met in het midden een stipellijn. Bovenaan de rechthoek zet je de voorwaarde waarop je brancht, met eronder de sequentie die doorgaat als de voorwaarde geld. In het tweede dele onder de stippellijn in de rechthoek zet je het alternatieve verloop met een [else] conditie. 

\section{Systeem Sequentie Diagram}
\begin{definition}
Een systeem sequentiediagram is een sequentiediagram dat alle interacties tussen actor(en) en het systeem van 1 use case scenario weergeeft. 
\end{definition}

Door deze beschrijving van de use case in een diagram te plaatsen kun je gemakkelijk inzicht krijgen in de systeemeisen die een eindgebruiker verwacht. Een Systeem Sequentie Diagram (SSD) is dus een speciaal SD die de interactie tussen de actor en het systeem toont voor 1 use case scenario. Een SSD toont dus de acties (systeemevents) die een externe actor genereert (incluis de orde en systeemoperaties die als gevolg van de systeemevents uitgevoerd worden) in een afbeelding. In een use case worden de verschillende stappen ook genummerd. Dit is belangrijk want die volgorde zal ook verticaal weergegeven worden in een SSD.

\subsection{Nut van SSD}
\begin{itemize}
	\item SSD geeft inzicht in de systeemeisen die een eindgebruiker verwacht
	\item SSD toont de systeem boodschappen/operaties = systeemgedrag
	\item SSD toont wat het systeem moet doen, niet hoe (black box)
	\item SSD bevat de systeemoperaties. Als een actor communiceert met het systeem gebeurt dit altijd via dergelijke systeemoperaties. Dus alle mogelijk communicatie die een actor kan hebben met het systeem, wordt voorgesteld door een systeemoperatie op een SSD
	\item Als we dan al die gevonden systeemoperaties proberen uit te werken (ontwerp + implementatie) dan krijgt onze software vorm. Het is dus eigenlijk een startpunt voor het ontwerp.
\end{itemize}

Aangeraden wordt om een SSD aan te maken voor het normale verloop en veel voorkomende of complexe alternatieve scenario's.

\subsection{Opstellen van een SSD via een UC}
\begin{enumerate}
	\item Kies een use case scenario. 
	\item Teken een levenslijn voor elke actor die in dit use case scenario een interactie heeft met het systeem. Benoem actor.
	\item Teken een rechthoek en bijhorende levenslijn die het systeem als een blackbox voorstelt
	\item Overloop gekozen use case scenario (van 1 \dots einde)
\begin{itemize}
	\item Voor elke actie die de actor uitvoert, dient een boodschap te worden toegevoegd aan het SSD. 
	\item Wat het systeem dan uitvoert is black box (is op SSD niet belangrijk). We tekenen enkel een activeringsblokje om aan te duiden dat het systeem een actie zal uitvoeren op vraag van de actor. 
	\item Als het systeem informatie teruggeeft aan de actor dan tekenen we een of meerdere terugkeerpijlen. Bovenop plaatsen we welke informatie de actor terugkrijgt.
\end{itemize}
\end{enumerate}

\subsubsection{Opmerkingen}
\begin{itemize}
	\item Duidelijke naamgeving voor de systeemboodschappen 
\begin{itemize}
	\item Moet duidelijk zeggen wat er gevraagd wordt aan het systeem, wat de bedoeling is dat het systeem doet. (Niet wat de actor doet!)
	\item Moet de bedoeling op een hoog abstractieniveau weergeven en niet in termen van inputtechnologie
	\item Sluit aan bij het gebruikte jargon in het domeinmodel
	\item Formuleert een opdracht \texttt{doeIets} ($\Rightarrow$ starten met de stam van een werkwoord)
	\item Gegevens die meegegeven worden aan het systeem worden tussen ronde haakjes weergegeven. We noemen dit parameters. Deze parameters zijn informatie die het systeem nodig heeft om zijn taak uit te voeren.
\end{itemize}
	\item Het systeem retourneert gegevens
	\item Het systeem geeft informatie aan de actor. Voor sommige informatie zijn gegevens nodig, voor andere niet.
	\item Als het systeem informatie uit het domein teruggeeft, dan tekenen we een pijl, anders niet

\end{itemize}

\section{Operation Contract}

Een operation contract is een onderdeel van de analyse die de veranderingen in het systeem gaat vastleggen ten gevolge van een systeemoperatie. We gaan voor elke systeemoperatie die een dergelijke verandering teweeg brengt een OC maken. Let op! Niet alle systeemoperaties hebben een gevolg op de staat van het systeem.

\subsection{Voorstelling}
Een operation  contract is een tabel die bestaat uit volgende zaken:
\begin{itemize}
	\item \textbf{Operation} : naam van betreffende operatie en de paramters
	\item \textbf{Cross References }: naam van de gerelateerde UCs
	\item \textbf{Precondities} : voorwaarden die voldaan moetne zijn voordat de operatie uitgevoerd wordt.
	\item \textbf{Postcondities} : gedetailleerde beschrijving van de gewijzigde toestand van de domeinobjecten na de uitvoering van de operatie. 
\end{itemize}

Een operation contract zet je dan in een vergelijkbare tabel:

\begin{table}[htbp]
	\centering
		\begin{tabular}{|l|p{9cm}|}
			\hline
			Contract & \\
			\hline
			Operation & \\
			\hline
			Cross References & \\
			\hline
			Precondities & \\
			\hline
			Postcondities & \\
			\hline
		\end{tabular}
	\caption{OC template}
	\label{tab:OCTemplate}
\end{table}

\subsection{Wanneer een OC maken}

Een operation contract stellen we dus op bij het veranderen van de staat van het systeem door de systeemoperatie. Er zijn drie manieren waarop dit kan gebeuren:
\begin{enumerate}
	\item Creatie van instantie/Verwijdering van een instantie
	\item Creatie van een associatie/Verbreking van een associatie
	\item Wijziging van attributen
\end{enumerate}

Wanneer dus na een systeemoperatie een van de bovengenoemde acties uitgevoerd geweest zijn, dan stellen we een operation contract op. Deze OC's zullen ons helpen bij het verdere ontwerp (wat er allemaal IN die zwarte doos gebeurt). Bij dat ontwerp moeten we ervoor zorgen dat we de beschreven postcondities bereiken.










\bibliographystyle{plain}
\bibliography{biblio}
\end{document}
